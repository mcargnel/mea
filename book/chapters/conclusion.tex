%# TODO: finish this section
In this document, we have explored the integration of Double Machine Learning (DML) techniques within the Difference-in-Differences (DiD) framework, highlighting their potential to enhance causal inference in econometric analyses. By leveraging the predictive capabilities of machine learning algorithms while maintaining the interpretability and rigor of traditional econometric methods, DML offers a promising avenue for researchers seeking to uncover causal relationships in complex data settings.

The application presented demonstrated differences in treatment effect estimates when comparing traditional DiD methods with DML-enhanced approaches. This underscores the importance of considering flexible modeling techniques to capture underlying data structures that may be overlooked by linear specifications.

However, it is important to note that while DML provides greater flexibility and potential for reduced bias, all causal inference methods are difficult to generalize outside the specific context of the study, as discussed in \cite{pritchett_2024}. Therefore, these methods are not by any means a replacement for a clear understanding of the phenomenon being studied, but rather a complement to better understand the underlying dynamics and improve the precision of causal estimates.