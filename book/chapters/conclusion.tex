In this thesis, we have explored the integration of Double Machine Learning (DML) techniques within the Difference-in-Differences (DiD) framework, assessing their potential to enhance causal inference in econometric analyses. By leveraging the predictive power of machine learning algorithms while preserving the interpretability of traditional econometric parameters, DML offers a robust alternative for researchers navigating complex, high-dimensional data settings.

Our empirical applications provided a comparative analysis of DML-DiD against traditional Two-Way Fixed Effects (TWFE) estimators. In the canonical single-period treatment setting (fracking regulation), both methods yielded nearly identical results. This suggests that when the data generating process is relatively simple and the set of controls is small, traditional linear specifications remain adequate. However, in the more complex staggered adoption setting (Castle Doctrine laws), the methods diverged significantly. While TWFE failed to detect a significant effect, the DML-DiD estimator uncovered a positive and statistically significant impact on homicide rates. This contrast highlights the specific value of DML: its ability to flexibly model nonlinear nuisance functions and handle complex treatment dynamics where rigid linear assumptions may fail.

These findings underscore that while DML is not a panacea, it is a critical tool for modern applied econometrics, particularly when theoretical guidance on functional forms is limited. Nevertheless, as noted by \cite{pritchett_2024}, methodological sophistication cannot substitute for external validity or a deep understanding of the institutional context. DML should therefore be viewed not as a replacement for economic intuition, but as a powerful complement that improves the precision and credibility of causal estimates within a well-defined identification strategy.